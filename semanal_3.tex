\documentclass{article}
\usepackage[utf8]{inputenc}
\usepackage{amsmath}
\usepackage{bussproofs} % Para reglas de inferencia
\usepackage{listings}
\usepackage{enumitem}

\title{Evaluación Semanal 3}
\author{Christian Eulogio Sánchez (320196872) \and Carlos Alonso Luna Rivera (318076814)}
\date{27 de agosto de 2024}

\begin{document}

\maketitle

\section*{Ejercicio 1}

Dadas las expresiones en sintaxis concreta de \textbf{MiniLisp}, vamos a resolver los tres apartados solicitados:

\subsection*{1.1 \texttt{(- (+ 20 3) (- -18 (+ 50 20)))}}

\begin{itemize}
    \item[(a)] \textbf{Sintaxis abstracta:}
    \begin{align*}
    \text{Sub}(
        &\text{Add}(\text{Const}(20), \text{Const}(3)), \\
        &\text{Sub}(\text{Const}(-18), \text{Add}(\text{Const}(50), \text{Const}(20)))
    )
    \end{align*}

    \item[(b)] \textbf{Evaluación usando reglas de semántica natural:}
    \begin{align*}
    \text{Sub}(
        &\text{Add}(20, 3),  \quad \text{(23)} \\
        &\text{Sub}(-18, \text{Add}(50, 20))  \quad \text{(-88)}
    )  \quad \text{Resultado: } 111
    \end{align*}
    
    \item[(c)] \textbf{Evaluación usando reglas de semántica estructural:}
    \begin{align*}
    &(- (+ 20 3) (- -18 (+ 50 20))) \\
    &(- 23 (- -18 70)) \\
    &(- 23 -88) \\
    &111 \quad \text{Resultado: } 111
    \end{align*}
\end{itemize}

\subsection*{1.2 \texttt{(not (+ 1 (- 3 (+ -8 1))))}}

\begin{itemize}
    \item[(a)] \textbf{Sintaxis abstracta:}
    \begin{align*}
    \text{Not}(
        &\text{Add}(\text{Const}(1), \text{Sub}(\text{Const}(3), \text{Add}(\text{Const}(-8), \text{Const}(1))))
    )
    \end{align*}

    \item[(b)] \textbf{Evaluación usando reglas de semántica natural:}
    \begin{align*}
    \text{Not}(
        &\text{Add}(1, \text{Sub}(3, \text{Add}(-8, 1))) \quad \text{(Resultado: } 11)
    ) \quad \text{Resultado: } \text{False}
    \end{align*}

    \item[(c)] \textbf{Evaluación usando reglas de semántica estructural:}
    \begin{align*}
    &(not (+ 1 (- 3 (+ -8 1)))) \\
    &(not (+ 1 (- 3 -7))) \\
    &(not (+ 1 10)) \\
    &(not 11) \\
    &\text{False} \quad \text{Resultado: } \text{False}
    \end{align*}
\end{itemize}

\subsection*{1.3 \texttt{(not (not (+ 3 5)))}}

\begin{itemize}
    \item[(a)] \textbf{Sintaxis abstracta:}
    \begin{align*}
    \text{Not}(
        &\text{Not}(\text{Add}(\text{Const}(3), \text{Const}(5)))
    )
    \end{align*}

    \item[(b)] \textbf{Evaluación usando reglas de semántica natural:}
    \begin{align*}
    \text{Not}(
        &\text{Not}(\text{Add}(3, 5)) \quad \text{(Resultado: } 8)
    ) \quad \text{Resultado: } \text{True}
    \end{align*}
    
    \item[(c)] \textbf{Evaluación usando reglas de semántica estructural:}
    \begin{align*}
    &(not (not (+ 3 5))) \\
    &(not (not 8)) \\
    &(not \text{False}) \\
    &\text{True} \quad \text{Resultado: } \text{True}
    \end{align*}
\end{itemize}

\section*{Ejercicio 2}

\subsection*{*}
\begin{enumerate}[label = (\alph*)]
    \item $<$ Expr $>$ ::= \ldots \\
          $|$  (* $<$Expr$>$ $<$Expr$>$)

    \item Representaremos la multiplicación con la etiqueta $Mult$.\\
    $\frac{i\text{ASA}\qquad j\text{ASA}}{Mult(i,j)\text{ASA}}$
    
    \item Paso Grande: \begin{center}
        $\frac{e\Rightarrow Num(n)\quad f\Rightarrow Num(n)}{Mult(e,f)\Rightarrow Num(n)}$
    \end{center}

    Paso Pequeño: \begin{center}
        $\frac{e\rightarrow e'}{Mult(e,f)\rightarrow Mult(e',f))}$
        $\frac{f\rightarrow f'}{Mult(Num(n),f)\rightarrow Mult(Num(n),f'))}$
        $\frac{}{Mult(Num(n_1),Num(n_2))\rightarrow Mult(n_1*n_2)}$
    \end{center}
\end{enumerate}

\subsection*{/}
\begin{enumerate}[label = (\alph*)]
    \item $<$ Expr $>$ ::= \ldots \\
          $|$ (/ $<$Expr$>$ $<$Expr$>$)
          $|$ $<\text{ String }>$

          $<\text{ String }>$ ::= \{$<\text{ A }>$\}\{$<\text{ a }>$\}

          $<\text{ A }>$ ::= A $|$ B $|$ C $|$ D $|$ E $|$ F $|$ G $|$ H $|$ I $|$ J $|$ K $|$ L $|$ M $|$ N $|$ O $|$ P $|$ Q $|$ R $|$ S $|$ T $|$ V $|$ W $|$ X $|$ Y $|$ Z $|$   $|$ $\epsilon$

          $<\text{ a }>$ ::= a $|$ b $|$ c $|$ d $|$ e $|$ f $|$ g $|$ h $|$ i $|$ j $|$ k $|$ l $|$ m $|$ n $|$ o $|$ p $|$ q $|$ r $|$ s $|$ t $|$ v $|$ w $|$ x $|$ y $|$ z $|$   $|$ $\epsilon$


    \item Representaremos la división con la etiqueta $Div$.\\
    $\frac{i\text{ASA}\qquad j\text{ASA}}{Div(i,j)\text{ASA}}$
    Y representaremos las cadenas con la etiqueta $String$.\\
    $\frac{s \in \Sigma^*}{String(s)\text{ASA}}$\\
    Donde $\Sigma$ = \{A-Z\}$\cup$\{a-z\}$\cup$\{" "\}
    
    \item Paso Grande: \begin{center}
        $\frac{f\Rightarrow Num(0)}{Mult(e,f)\Rightarrow String(s)}$\\
        $\frac{e\Rightarrow Num(n)\quad f\Rightarrow Num(n)}{Div(e,f)\Rightarrow Num(n)}$
    \end{center}
    
    Paso Pequeño: \begin{center}
        $\frac{f\rightarrow Num(0)}{Mult(e,f)\rightarrow String(s)}$
        $\frac{e\rightarrow e'}{Div(e,f)\rightarrow Div(e',f))}$
        $\frac{f\rightarrow f'}{Div(Num(n),f)\rightarrow Div(Num(n),f'))}$
        $\frac{}{Div(Num(n_1),Num(n_2))\rightarrow Div(n_1/n_2)}$
    \end{center}
\end{enumerate}


\subsection*{add1}
\begin{enumerate}[label = (\alph*)]
    \item $<$ Expr $>$ ::= \ldots \\
          $|$ (add1 $<$Expr$>$)
    
    \item Representaremos add1 con la etiqueta $Add1$.\\
    $\frac{i\text{ASA}}{Add1(i)\text{ASA}}$
    
    \item Paso Grande: \begin{center}
        $\frac{e\Rightarrow Num(n)}{Add1(e)\Rightarrow Num(n)}$
    \end{center}
    
    Paso Pequeño: \begin{center}
        $\frac{e\rightarrow e'}{Add1(e)\rightarrow Add1(e')}$
        $\frac{e\rightarrow Num(n)}{Add1(e)\rightarrow Add(n,1)}$
    \end{center}
\end{enumerate}

\subsection*{sub1}
\begin{enumerate}[label = (\alph*)]
    \item $<$ Expr $>$ ::= \ldots \\
          $|$ (sub1 $<$Expr$>$)

    
    \item Representaremos sub1 con la etiqueta $Sub1$.\\
    $\frac{i\text{ASA}}{Sub1(i)\text{ASA}}$

    \item Paso Grande: \begin{center}
        $\frac{e\Rightarrow Num(n)}{Sub1(e)\Rightarrow Num(n)}$
    \end{center}
    
    Paso Pequeño: \begin{center}
        $\frac{e\rightarrow e'}{Sub1(e)\rightarrow Sub1(e')}$
        $\frac{e\rightarrow Num(n)}{Sub1(e)\rightarrow Sub(n,1)}$
    \end{center}
\end{enumerate}

\subsection*{sqrt}
\begin{enumerate}[label = (\alph*)]
    \item $<$ Expr $>$ ::= \ldots \\
          $|$ (sqrt $<$Expr$>$)

    
    \item Representaremos la raíz cuadrada con la etiqueta $Sqrt$.\\
    $\frac{i\text{ASA}}{Sqrt(i)\text{ASA}}$

    \item Paso Grande: \begin{center}
        $\frac{e\Rightarrow Num(n)<Num(0)}{Sqrt(e)\Rightarrow String(s)}$
        $\frac{e\Rightarrow Num(n)}{Sqrt(e)\Rightarrow Num(n)}$
    \end{center}
    
    Paso Pequeño: \begin{center}
        $\frac{f\rightarrow Num(n)<Num(0)}{Sqrt(e)\rightarrow String(s)}$
        $\frac{e\rightarrow e'}{Sqrt(e)\rightarrow Sqrt(e')}$
        $\frac{}{Sqrt(Num(n))\rightarrow sqrt(n)}$
    \end{center}
\end{enumerate}


\end{document}
